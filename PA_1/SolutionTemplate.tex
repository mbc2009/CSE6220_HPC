\documentclass[11pt]{article}
\oddsidemargin=-0.005in
\evensidemargin=-0.005in
\textwidth=6.50in
\textheight=9.0in
\topmargin=-0.5in
\setlength{\parindent}{0pt}
\newenvironment{tempspacing}[1]{\begin{trivlist}
\baselineskip=#1
\normalbaselineskip=\baselineskip\item[]}{\end{trivlist}}
\renewcommand{\baselinestretch} {1.2}
\pagenumbering{arabic}
\pagestyle{empty}
\thispagestyle{empty}

\begin{document}
%%% Frontmatter
\title{\large\bf \vspace*{-0.3in}CSE 6220 High Performance Computing\\
Programming Assignment-1 \\Submitted \today}
\author{}
\date{}
\maketitle
\vspace*{-0.8in}
{\bf Name:} Honglin Liu \hfill{{\bf GT login:} hliu686}\\

%%% Main text
A team contract is an agreement between you and your teammate(s) to establish a set of conventions about how your team will operate. 
Ideally, it ensures a smooth team experience — think back to good or bad aspects of past team projects when negotiating it.\\

Your contract should answer questions such as the following:
\begin{enumerate}
	\item How will you communicate outside of meetings (Zoom, phone, email, Slack, etc.)?
	\item How will you manage your code? (git, SVN, etc.)?
	\item If someone on the team decides to drop the class, what obligations do they have to their teammates?
	\item How will work be divided among team members?
	\item What expectations regarding team grade do team members have? If there are differences, is it acceptable for some team members to do more or less work than the others?
\end{enumerate}

If your team has problems working together, and you have little in your team contract, it will be more difficult for the course staff to help you. Your TA will review your team contracts and let you know of any concerns we see.









\end{document}

%%% Local Variables:
%%% mode: latex
%%% TeX-master: t
%%% End:
